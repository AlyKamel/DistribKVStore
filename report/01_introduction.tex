\section{Introduction}
\label{sec:introduction}

To develop a key-value database, one of the main properties to consider is, how to handle large amount of data while making the system available for multiple clients. As an example, a popular key-value database, DynamoDB which is used by Amazon, can be given. 

Amazon DynamoDB is a non-relational database system, which provides low latency, to be more specific single-digit millisecond latency\cite{amazon}, considering the system is used all over the world. Data records are stored with a primary or composite key and several attributes depending on client's request. The only constraint is that these records, including attribute names and attribute values, cannot exceed 400 KB which is rather a large space for storing the provided data\cite{amazon}.

Apache Cassandra is another popular key-value based database system, which is known for managing some of the world's largest datasets on clusters with the help of thousands of nodes distributed amongst multiple data centres. The distributed database system focuses on high level of availability by scaling millions of read and write requests per second. System handles the network partitions gracefully, in cases such as, single node failure or even entire data set errors\cite{chebotko2015big}.

Another example Redis --> TODO \cite{paksula2010persisting}

By observing the provided features of most commonly used databases, one can say that our program is a very small scaled version of them. It provides the core functionalities such as responding the client with consistent values as fast as possible, being available within a certain amount of time, and maintaining resilience towards partitions in the network.

Cloud Data Bases Praktikum started off with a simple implementation of a client with the aim of a better understanding of the client/server architecture and socket programming. Nonetheless, the system is developed with implementing a one-to-many relationship being one server and multiple clients. The program is improved with a basic key-value storage service extension in order for clients to perform read and write operations with the help of mentioned storage server. 

In the following milestones, the architecture is enhanced to a many-to-many relationship consisting of multiple servers and multiple clients. To manage the servers an External Configuration Service is created, thus the servers which are placed on an hashring can communicate with each other. Following that replication strategy is added to the system in order to maintain availability, consistency and partition tolerance. 

In the last step, namely milestone 5, we extended our program with a groupchat feature. In the decision process, our initial goal was to come up with an idea that the clients can make the most use of.

We started with examining the features of above-mentioned database systems to get inspired. Thereafter, we opted to have a chat system, where multiple clients can enter a chatroom and exchange messages. All messages send in the same room will be sent to all active members of that chatroom. The essence of providing chatrooms for clients is to retrieve information in a short amount of time by means of being able to ask another member of the same chatroom. 

Going into more detail, we decided to have two different chatroom options, being public and private rooms. In a public room clients can exchange more general information, such as TODO. In contrast, private rooms are suitable for sharing sensitive information, such as credit card passwords. Since a private chatroom can only be accessed via a password, clients are ensured to keep their messages safe.

Since clients can perform read and write operations while chatting, we wanted to have a chatbot to handle such requests. All messages, including the commands put and get, are directed to the chatbot which is responsible to access to the database and perform the requests.
 
This report focuses on giving a detailed explanation of our extension and discussing the outcomes of the performance analysis.
