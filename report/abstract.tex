\begin{abstract} 

In today's world, the key to retrieve substantial information is communication. Software that enables messaging features, such as calling and video transmission as in Skype, sending emails with Google Mail, chatting via WhatsApp or Facebook and many more, ease the exchange of information and are therefore crucial for productive working in groups of many people. They are used for managing, discussing, preparing, executing an organizing parts of a project. On this basis we dedicated us in milestone 5 to implement a feature that enables multiple clients to communicate via chatrooms.

This report begins with an introduction part, where different commercial databases are examined. Depending on our observations of how the commonly used database systems work, our motivation of developing the group chat extension is elaborated. 

In the background section a brief description of key-value stores and caching is given. Moving on, CAP Theorem, ACID and BASE design approaches are clarified for the sake of a better understanding of the developed distributed and replicated database system.

Furthermore, the milestone 4 is revisited by explaining the architecture of our key-value server. The client and the server sides are discussed respectively.

Followed by the revisit of milestone 4, the group chat extension is examined in depth. The basic functionalities of the group chat and the implementation of the chat system are studied in greater detail. Nevertheless, the database access during a chat session is clarified.

Eventually, we conclude this paper with a performance analysis and depending on the results, a summary about the advantages and the disadvantages of our system is elaborated.

\end{abstract}



