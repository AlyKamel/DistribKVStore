\begin{abstract}

In today's world, communication is key between people to retrieve information, which is substantial. Almost every software system includes a messaging platform nowadays in order to ease their clients' lives by offering a chat service that enables sharing essential information. On this basis we extended our program with a group chat feature where multiple clients can communicate with each other via chatrooms.  

In today’s world, the key to retrieve substantial information is communication. Software that enables messaging features, such as calling and video transmission (e.g. Skype), mailing (e.g. Google Mail), chatting (e.g. WhatsApp, Facebook) and many more, ease the exchange and are therefore crucial for productive working in groups of many people. They are typically used for managing, discussing, preparing, executing or organizing parts of the work or project. On this basis we dedicated us in MS5 (milestone 5) to implement a feature that enables multiple clients to communicate via chat rooms.

%TODO: this should be longer

This report begins with an introduction part, where different commercial databases are examined. Depending on our observations of how the commonly used database systems work, our motivation of developing the group chat extension is elaborated. 

Moving on, the background containing CAP Theorem, ACID and BASE design approaches are clarified for the sake of a better understanding of the developed distributed and replicated database system.

Furthermore, our extension is examined focusing mainly on how the system works from the client's point of view. The features of the group chat and the implementation are studied in greater detail. 
In the implementation section a brief description of the replicated and distributed storage service from Milestone 4 is represented. The replication strategy is an important feature of database systems in general because it guarantees eventual consistency and basic availability which is crucial for distributed database systems.

The main idea with the group chat extension is that multiple clients can join a chatroom with a globally unique chatID, assigned individually for each room, and exchange messages there. Moreover, clients can perform read and write operations with the help of a chatbot while being in the chatroom.

Eventually we conclude our paper with a performance analysis by comparing results from our performance measurement tests and touch upon the advantages of providing a group chat.
\end{abstract}



