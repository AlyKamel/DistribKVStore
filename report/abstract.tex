\begin{abstract}

In today's world, communication is key between people to retrieve information that is substantial. Almost every software system includes a messaging platform nowadays in order to ease their client’s lives by offering a chat service where essential information can be shared. On this basis we extended our program with a group chat feature, where multiple clients can communicate with each other via chatrooms.  %TODO: this should be longer

This report begins with an introduction part, where different databases are examined. Depending on our observations of how the commonly used database systems work, our motivation of developing the groupchat extension is elaborated. 

Moving on, the background containing CAP Theorem, ACID and BASE design approaches are clarified for the sake of a better understanding of the developed distributed and replicated database system.

Furthermore, our extension is examined focusing mainly on how the system works from the client's point of view. The features of the group chat and the implementation is studied in greater detail. 
In the implementation section a brief description of the replicated and distributed storage service from milestone 4 is represented as well. The replication strategy is an important feature of database systems in general because it guarantees eventual consistency and basic availability, which is crucial for distributed database systems.

The main idea with the groupchat extension is that multiple clients can join a chatroom with a chatID, assigned individually for every different room, and exchange messages there. Moreover, clients can perform read and write operations with the help of a chatbot while being in the chatroom.

Eventually we conclude our paper with a performance analysis by comparing results from our performance measurement tests and touch upon the advantages of providing a group chat.
\end{abstract}



