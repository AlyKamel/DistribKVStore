\section{Background}
\label{sec:background}
One of the main goals implementing milestone 4 and milestone 5 is to have a system which is preferable for the clients by designing it with the following features: high availability, eventual consistency and partition tolerance.

% TODO change enumeration method 
\subsection{CAP Theorem}
\label{sec:backgorund_cap}  % TODO change citation
According to the CAP Theorem, distributed database systems have three substantial properties to consider which stand for the name of the theorem itself: consistency, availability and partition tolerance. Eric Brewer, the founder of the CAP theorem, aimed to clarify with this theorem that a distributed database system can have at most two of the three mentioned properties. Reference: E. Brewer, "Towards Robust Distributed Systems

\subsubsection{Consistency}
\label{sec:backgorund_cap_consistency}
When a system is focusing on consistency, clients should be provided with the most up-to-date data, meaning the fresh data that after the last write operation.

\subsubsection{Availability}
\label{sec:backgorund_cap_avaiability}
"A" in CAP referring to the property availability, centres upon responding the request of the client in any cases.

\subsubsection{Partition tolerance}
\label{sec:backgorund_cap_partitiontolarence}
Partitions are failures which can be encountered in distributed database systems, namely crashed servers or dropped packets. Lastly, if a system maintains partition tolerance, the system will handle the problem without having to shut down.\\
\\
%TODO change citations
Later on, Brewer mentions in his article "Cap 12 Years Later: How the Rules Have Changed" that designers do not have to abide strictly to the 2 of 3 principle, it is rather a spectrum than binary options. In other words, a distributed database system can favour high level of consistency and partition tolerance by having low level of availability. Thus, the initial theorem is improved by not having to sacrifice availability completely. Reference:Cap twelve years later

\subsection{BASE}
\label{sec:backgorund_base}
BASE is one of the design approaches in terms of the CAP theorem which is also created by Eric Brewer. Non-relational database systems(NoSql) make use of the BASE approach, which concentrates on high availability. BASE is an acronym which stands for basic availability, soft state and eventual consistency. With milestone 4, replication is added to the system with the intention of increasing the availability by distributing the data records to the two replica servers.

\subsubsection{Basic Availability}
\label{sec:backgorund_base_basicavailability}
Client is guaranteed to get a response from the servers in a matter of time, but the returned value may be stale, stated in other words not consistent, since the system is focusing on high availability.
\subsubsection{Soft State}
\label{sec:backgorund_base_softstate}
Without any updates state changes can be observed within the system.
\subsubsection{Eventual Consistency}
\label{sec:backgorund_base_eventualconsistency}
If there are no updates in the system for a long time, then all servers will gradually become consistent. reference: Milestone4 slides\\
\\
Implementing the program, we paid attention that the system we created, provided the above-mentioned properties on a high scale.\\Our system contains the first property of BASE being basic availability, because servers will respond to request of the clients even if a latency occurs. Since single failing nodes can be tolerated due to replication, partition tolerance is provided as well. At last, eventual consistency is achieved within the system. If there are no updates for a long time all the replicas will become eventually consistent.

    


