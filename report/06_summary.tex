\section{Summary}
\label{sec:summary}

To conclude our paper, with the group chat extension in our system we provided clients a chat system which supports low latency and high availability. We got inspired by Redis Pub/Sub strategy and adjusted it in our system by bringing the clients on the same level instead of having publishers and subscribers. As we mentioned in our motivation \ref{sec:motivation} subscribers are waiting on an update of the subscribed key and during this process clients do not use any other functionalities in the system. To prevent the idle time which client just waits a response from the server, we enhanced the system maintaining many-to-many relationship in contrast Redis’ one-to-many relationship.

During the development process, initially we considered the demands of the client, for instance having globally unique usernames, accessing the database while chatting and having private and public chatrooms. Subsequently, with the implementation of the chat system we fulfilled the potential demands and extent it with features such as “ACTIVE” or “WSP” commands as explained in the section Newly added commands for the group chat \ref{sec:groupchat_commands}.

Consequently, after the performance tests we recommend our system for light-weighted communication. The reason for that is as explained in the Server side implementation \ref{sec:implementation_serverside} in greater detail, it might occur that the chatrooms are not equally distributed in the system and thereby just one server might have to carry the load of having all the chatrooms. And this leads us one of our so to say weaknesses in the system which is the limitations. We limited the number of chatrooms to 15 for one server and the number of users using one chatroom to 30. The numbers could be arranged depending on the use case of the system. If the user of our group chat prefers to work on a large-scaled messaging platform and can give up on having low latency, the limitations for the number of chatrooms and the clients could be increased.

One of our strengths is allowing the access to the database while clients are chatting.

