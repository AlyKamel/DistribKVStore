%%
%% This is file `sample-sigconf.tex',
%% generated with the docstrip utility.
%%
%% The original source files were:
%%
%% samples.dtx  (with options: `sigconf')
%% 
%% IMPORTANT NOTICE:
%% 
%% For the copyright see the source file.
%% 
%% Any modified versions of this file must be renamed
%% with new filenames distinct from sample-sigconf.tex.
%% 
%% For distribution of the original source see the terms
%% for copying and modification in the file samples.dtx.
%% 
%% This generated file may be distributed as long as the
%% original source files, as listed above, are part of the
%% same distribution. (The sources need not necessarily be
%% in the same archive or directory.)
%%
%% The first command in your LaTeX source must be the \documentclass command.
\documentclass[sigconf]{acmart}

%%
%% Settings and packages

%% Packages
\usepackage[acronym,toc,nonumberlist]{glossaries}
\usepackage[inline]{enumitem}
\usepackage[anythingbreaks]{breakurl}

%%
%% Hyphenation rules adding a bit more human intel to stupid latex
\hyphenation{data-bases}

%%
%% Commands for first letter bold
\newcommand{\fb}[1]{\dofb#1}
\newcommand{\dofb}[1]{\textbf{#1}\nobreak\hspace{0pt}}

%%
%% The majority of ACM publications use numbered citations and
%% references.  The command \citestyle{authoryear} switches to the
%% "author year" style.
%%
%% If you are preparing content for an event
%% sponsored by ACM SIGGRAPH, you must use the "author year" style of
%% citations and references.
%% Uncommenting
%% the next command will enable that style.
%% \citestyle{acmauthoryear}

%%
%% Glossaries
\makeglossaries
%% Use TOC dots. Not used anymore, as page numbers were killed
%% \renewcommand*\glspostdescription{\cftdotfill{\cftdotsep}}
%% Add support for tree glossaries
%% \renewcommand{\glossarypreamble}{\glsfindwidesttoplevelname[\currentglossary]}
%% Number of columns for mcols
%% \renewcommand{\glsmcols}{2}
\newacronym{dbms}{DBMS}{database management system}
\newacronym[shortplural={DDBS}]{ddbs}{DDBS}{distributed database system}


%%
%% Commalist
\newlist{commalist}{enumerate*}{1}
\setlist[commalist]{
    label=(\arabic*),
    itemjoin={{, }},
    itemjoin*={{, and }},
    afterlabel=\unskip{{~}}
}

\newlist{commalistor}{enumerate*}{1}
\setlist[commalistor]{
    label=(\arabic*),
    itemjoin={{; }},
    itemjoin*={{; or }},
    afterlabel=\unskip{{~}}
}

%%
%% Break URLs on dashes
\def\UrlBreaks{\do\/\do-}


%%
%% end of the preamble, start of the body of the document source.
\begin{document}

%%
%% The "title" command has an optional parameter,
%% allowing the author to define a "short title" to be used in page headers.
\title{Cloud database}

%%
%% Get authors from external file

\author{Aly Kamel}
\authornote{Authors contributed equally to this work.}
\email{aly.kamel@tum.de}
\affiliation{%
\institution{Technical University of Munich}
}
\author{Iremnur Kidil}
\authornotemark[1]
\email{ge34hoz@tum.de}
\affiliation{%
\institution{Technical University of Munich}
}
\author{Ricardo Kraft}
\authornotemark[1]
\email{...}
\affiliation{%
\institution{Technical University of Munich}
}



\renewcommand{\shortauthors}{Kamel}
\renewcommand{\shortauthors}{Kidil}
\renewcommand{\shortauthors}{Kraft}

%%
%% The abstract is a short summary of the work to be presented in the
%% article.
% !TeX root = ../main.tex

% TODO: outline

\begin{abstract}
    Today, almost
\end{abstract}


%%
%% The code below is generated by the tool at http://dl.acm.org/ccs.cfm.
%% Please copy and paste the code instead of the example below.
%%
\begin{CCSXML}
	<ccs2012>
	<concept>
	<concept_id>10010520.10010553.10010562</concept_id>
	<concept_desc>Computer systems organization~Embedded systems</concept_desc>
	<concept_significance>500</concept_significance>
	</concept>
	<concept>
	<concept_id>10010520.10010575.10010755</concept_id>
	<concept_desc>Computer systems organization~Redundancy</concept_desc>
	<concept_significance>300</concept_significance>
	</concept>
	<concept>
	<concept_id>10010520.10010553.10010554</concept_id>
	<concept_desc>Computer systems organization~Robotics</concept_desc>
	<concept_significance>100</concept_significance>
	</concept>
	<concept>
	<concept_id>10003033.10003083.10003095</concept_id>
	<concept_desc>Networks~Network reliability</concept_desc>
	<concept_significance>100</concept_significance>
	</concept>
	</ccs2012>
\end{CCSXML}

\ccsdesc[500]{Computer systems organization~Embedded systems}
\ccsdesc[300]{Computer systems organization~Redundancy}
\ccsdesc{Computer systems organization~Robotics}
\ccsdesc[100]{Networks~Network reliability}

%%
%% Keywords. The author(s) should pick words that accurately describe
%% the work being presented. Separate the keywords with commas.
\keywords{datasets, neural networks, gaze detection, text tagging}

%%
%% This command processes the author and affiliation and title
%% information and builds the first part of the formatted document.
\maketitle

%%
%% Main chapters. Feel free to add as many as you wish.
\section{Introduction}
\label{sec:introduction}

One of the main criteria when developing a key-value database(DB), is being able to handle large amounts of data while making the system available for multiple clients. DB systems, such as AmazonWeb Services, Microsoft Azure SQL and Oracle Database offer the combination of scalability and cost effectiveness.

Another important feature is elasticity which measures the ability of a system to dynamically scaled up or down. The key-value DB BigTable and PNUTS are great examples\cite{agrawal2011database}.

Amazon DynamoDB is a non-relational database system which provides single-digit millisecond latency \cite{amazon}, considering the system is used worldwide. Data records are stored with a primary or composite key and several attributes depending on client's request \cite{kalid2017big}. 

Apache Cassandra is another key-value based high scaled \cite{abadi2012consistency} DB system, which is known for managing some of the world's largest datasets on clusters with the help of thousands of nodes distributed amongst multiple data centres \cite{chebotko2015big}. Cassandra offers more flexibility regarding fault handling and managing wide range data, unlike BigTable and Dynamo \cite{kalid2017big}. The system focuses on high level of availability by scaling millions of read and write requests per second\cite{chebotko2015big}.

Our inspiration mainly came from the DB Redis. It rose in popularity after its creation in 2009 and got deployed by many big companies such as Instagram \cite{krieger2011instagram} and Twitter \cite{yu2014twitter}, which require enormous DBs and fast responses to the millions of users they serve. One of its multiple features is the Pub/Sub system \cite{redis2020pubsub}, added in March 2010 \cite{sanfilippo2010pubsub}. With it, clients are able to subscribe to a set of keys, namely topics, and receive an update whenever their associated values get overwritten. In this model, clients keep waiting for the server to send them an update. During that period, subscribers are expected to use only subscription related operations.

In our system, we wanted for there to be no distinction between publishers and subscribers with everyone being able to both publish and receive updates. Apart from that, access to the DB should still be possible. 

Thus, the way we modelled our system in the end, was having a many-to-many relationship of all subscribed clients, effectively, implementing a group chat functionality.



\section{Background}
\label{sec:background}
In this section, we elaborate on some of the basic concepts that our key-value system depends on. First, we discuss the CAP theorem and analyse where our system lies on that spectrum. Then, we mention two design paradigms, namely BASE and ACID, and explain the differences between them.

\subsection{Key-value store}
Key-value stores are perhaps the simplest type of a NoSQL DB. Every information is saved in the form of key value pairs. Typically the key’s memory space is very small compared to the one of the values. The key is used as an identifier to the actual data saved in the value. The client will always need to have a key in order to work with the DB. 

In our case values are limited to strings and can be easily stored and retrieved just by providing the corresponding key, which is also limited to a string.


\subsection{Persistence}
%TODO add how data gets stored on disk

\subsection{Caching}
write through policy

\subsection{CAP Theorem}
\label{sec:background_cap} 
According to the CAP Theorem, distributed databases have three substantial properties to consider: consistency, availability and partition tolerance\cite{brewer2012cap}. Eric Brewer, the man behind the CAP theorem, stated that a distributed database can fulfil at most two of three properties\cite{brewer2000cap}:

\begin{itemize}
  \item Consistency: \\
  Clients are provided with fresh data, meaning the most up-to-date version of the data that got stored after the last write operation.
  \item Availability: \\
  Servers respond to the request of clients at all cases. Every non-failing node in the system must be able to serve the client in a reasonable amount of time\cite{gilbert2002brewer}.
  \item Partition tolerance: \\
  Whenever a server crashes, the rest of the system will still stay functional after the crash, without any information being lost.
\end{itemize}

Twelve years after from proposing the CAP theorem, Brewer mentioned that software engineers do not have to strictly abide to the 2 of 3 principle, it is rather a spectrum than a binary choice \cite{brewer2012cap}. In other words, a distributed DBS can favour high level of consistency and partition tolerance by having low level of availability. Thus, the initial theorem is improved by not having to sacrifice availability completely in this case.

There are two design approaches for distributed database systems, namely ACID and BASE. According to Brewer, these two design approaches may be referred as opposites of each other\cite{brewer2012cap} because of their priorities and use cases. ACID approach is used most of the times for relational database systems (SQL) and focuses on consistency to maintain reliability. On the contrary BASE is more suitable for the non-relational database systems (NoSQL) concentrating on providing the client high level of availability\cite{brewer2000cap}.

\subsection{ACID}
\label{sec:background_acid}
ACID is a traditional design approach\cite{brewer2012cap} when it comes to large-scaled distributed systems. The main goal of ACID is that despite the system having partitions, the client should always be provided with consistent values. It is an acronym which stands for the four following properties:
\begin{itemize}
\item Atomicity:\\
A set of transactions succeed all at once or fail all together. In other words, it is an all or nothing strategy.
\item Consistency: \\
The system never contains any stale data. When a client wants to read from any server, the returned value must be the value from the last write commit by any client. All clients should always get the same result whenever they try to fetch the same information.
\item Isolation:\\
All transactions happen isolated from each other and do not affect each other. Thus, if one transaction fails, all others are undisturbed.
\item Durability:\\
Once a client is informed that a transaction has been successfully committed, even if a crash occurs, the transaction would still stay committed.
\end{itemize} 

\subsection{BASE}
\label{sec:background_base}
BASE, another design approach defined by Brewer, offers looser requirements than ACID\cite{brewer2012cap}. Non-relational database systems concentrating on high availability make use of the BASE approach, which sacrifices strong consistency in favour of being always accessible. Examples include huge storage services such as Amazon's Dynamo, Facebook's Cassandra or Google's BigTable, where millions of active users always expect the service to be available\cite{kalid2017big}. Nevertheless, there must be trade-offs. By not guaranteeing the consistency at all time, the system cost is reduced, and clients are happier, but a client might not get an immediate response to his request.

\begin{itemize}
  \item Basic Availability:\\
 Clients are guaranteed to get a quick response from the server without getting blocked, however the returned value may be stale, stated in other words, inconsistent with the latest version.
  \item Soft State:\\
Since stale data is permitted, servers never truly know whether they are currently up-to-date or if they still contain invalid data.
  \item Eventual Consistency:\\
If there are no updates in the system for a long time, then all servers will gradually become consistent. Since the time is not specified, servers are always in a soft state, as explained above.
\end{itemize}

    


\section{Summary}
\label{sec:summary}

To conclude our paper, with the group chat extension in our system we provided clients a chat system which supports low latency and high availability. We got inspired by Redis Pub/Sub strategy and adjusted it in our system by bringing the clients on the same level instead of having publishers and subscribers. As we mentioned in the motivation section \ref{sec:motivation} subscribers are waiting on an update of the subscribed key and during this process clients do not use any other functionalities in the system. To prevent the idle time which client just waits a response from the server, we enhanced the system maintaining many-to-many relationship in contrast Redis’ one-to-many relationship.

During the development process, initially we considered the demands of the client, for instance having globally unique usernames, accessing the database while chatting and having private and public chatrooms. Subsequently, with the implementation of the chat system we fulfilled the potential demands and extent it with features such as “ACTIVE” or “WSP” commands as explained in the section Newly added commands for the group chat \ref{sec:groupchat_commands}.

Consequently, after the performance tests we recommend our system for light-weighted communication. The reason for that is as explained in the Server side implementation \ref{sec:implementation_serverside} in greater detail, it might occur that the chatrooms are not equally distributed in the system and thereby just one server might have to carry the load of having all the chatrooms. And this leads us one of our so to say weaknesses in the system which is the limitations. We limited the number of chatrooms to 15 for one server and the number of users using one chatroom to 30. The numbers could be arranged depending on the use case of the system. If the user of our group chat prefers to work on a large-scaled messaging platform and can give up on having low latency, the limitations for the number of chatrooms and the clients could be increased.

One of our strengths is allowing the access to the database while clients are chatting.



%%
%% The next two lines define the bibliography style to be used, and
%% the bibliography file.
\bibliographystyle{ACM-Reference-Format}
\bibliography{bibliography}

%%
%% If your work has an appendix, this is the place to put it.
\appendix


\end{document}
\endinput
