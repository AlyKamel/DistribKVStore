%%
%% This is file `sample-sigconf.tex',
%% generated with the docstrip utility.
%%
%% The original source files were:
%%
%% samples.dtx  (with options: `sigconf')
%% 
%% IMPORTANT NOTICE:
%% 
%% For the copyright see the source file.
%% 
%% Any modified versions of this file must be renamed
%% with new filenames distinct from sample-sigconf.tex.
%% 
%% For distribution of the original source see the terms
%% for copying and modification in the file samples.dtx.
%% 
%% This generated file may be distributed as long as the
%% original source files, as listed above, are part of the
%% same distribution. (The sources need not necessarily be
%% in the same archive or directory.)
%%
%% The first command in your LaTeX source must be the \documentclass command.
\documentclass[sigconf]{acmart}

%%
%% Settings and packages

%% Packages
\usepackage[acronym,toc,nonumberlist]{glossaries}
\usepackage[inline]{enumitem}
\usepackage[anythingbreaks]{breakurl}
\usepackage{subcaption}

%%
%% Hyphenation rules adding a bit more human intel to stupid latex
\hyphenation{data-bases}

%%
%% Commands for first letter bold
\newcommand{\fb}[1]{\dofb#1}
\newcommand{\dofb}[1]{\textbf{#1}\nobreak\hspace{0pt}}

%%
%% The majority of ACM publications use numbered citations and
%% references.  The command \citestyle{authoryear} switches to the
%% "author year" style.
%%
%% If you are preparing content for an event
%% sponsored by ACM SIGGRAPH, you must use the "author year" style of
%% citations and references.
%% Uncommenting
%% the next command will enable that style.
%% \citestyle{acmauthoryear}

%%
%% Glossaries
\makeglossaries
%% Use TOC dots. Not used anymore, as page numbers were killed
%% \renewcommand*\glspostdescription{\cftdotfill{\cftdotsep}}
%% Add support for tree glossaries
%% \renewcommand{\glossarypreamble}{\glsfindwidesttoplevelname[\currentglossary]}
%% Number of columns for mcols
%% \renewcommand{\glsmcols}{2}
\newacronym{dbms}{DBMS}{database management system}
\newacronym[shortplural={DDBS}]{ddbs}{DDBS}{distributed database system}

%%
%% Commalist
\newlist{commalist}{enumerate*}{1}
\setlist[commalist]{
    label=(\arabic*),
    itemjoin={{, }},
    itemjoin*={{, and }},
    afterlabel=\unskip{{~}}
}

\newlist{commalistor}{enumerate*}{1}
\setlist[commalistor]{
    label=(\arabic*),
    itemjoin={{; }},
    itemjoin*={{; or }},
    afterlabel=\unskip{{~}}
}

%%
%% Break URLs on dashes
\def\UrlBreaks{\do\/\do-}


%%
%% end of the preamble, start of the body of the document source.
\begin{document}

%%
%% The "title" command has an optional parameter,
%% allowing the author to define a "short title" to be used in page headers.
\title{Cloud database}

%%
%% Get authors from external file

\author{Aly Kamel}
\authornote{Authors contributed equally to this work.}
\email{aly.kamel@tum.de}
\affiliation{%
\institution{Technical University of Munich}
}
\author{Iremnur Kidil}
\authornotemark[1]
\email{ge34hoz@tum.de}
\affiliation{%
\institution{Technical University of Munich}
}
\author{Ricardo Kraft}
\authornotemark[1]
\email{...}
\affiliation{%
\institution{Technical University of Munich}
}



\renewcommand{\shortauthors}{Kamel}
\renewcommand{\shortauthors}{Kidil}
\renewcommand{\shortauthors}{Kraft}

%%
%% The abstract is a short summary of the work to be presented in the
%% article.
\begin{abstract} 

In today's world, the key to retrieve substantial information is communication. Software that enables messaging features, such as calling and video transmission as in Skype, sending emails with Google Mail, chatting via WhatsApp or Facebook and many more, ease the exchange of information and are therefore crucial for productive working in groups of many people. They are used for managing, discussing, preparing, executing an organizing parts of a project. On this basis we dedicated us in milestone 5 to implement a feature that enables multiple clients to communicate via chatrooms.

This report begins with an introduction part, where different commercial databases are examined. Depending on our observations of how the commonly used database systems work, our motivation of developing the group chat extension is elaborated. 

In the background section a brief description of key-value stores and caching is given. Moving on, CAP Theorem, ACID and BASE design approaches are clarified for the sake of a better understanding of the developed distributed and replicated database system.

Furthermore, the milestone 4 is revisited by explaining the architecture of our key-value server. The client and the server sides are discussed respectively.

Followed by the revisit of milestone 4, the group chat extension is examined in depth. The basic functionalities of the group chat and architecture of the chat system are studied in greater detail. Nevertheless, the database access during a chat session is clarified afterwards.

Eventually, the report is concluded with a performance analysis and depending on the results, a summary about the advantages and the disadvantages of our system is elaborated.

\end{abstract}





%%
%% The code below is generated by the tool at http://dl.acm.org/ccs.cfm.
%% Please copy and paste the code instead of the example below.
%%
\begin{CCSXML}
	<ccs2012>
	<concept>
	<concept_id>10010520.10010553.10010562</concept_id>
	<concept_desc>Computer systems organization~Embedded systems</concept_desc>
	<concept_significance>500</concept_significance>
	</concept>
	<concept>
	<concept_id>10010520.10010575.10010755</concept_id>
	<concept_desc>Computer systems organization~Redundancy</concept_desc>
	<concept_significance>300</concept_significance>
	</concept>
	<concept>
	<concept_id>10010520.10010553.10010554</concept_id>
	<concept_desc>Computer systems organization~Robotics</concept_desc>
	<concept_significance>100</concept_significance>
	</concept>
	<concept>
	<concept_id>10003033.10003083.10003095</concept_id>
	<concept_desc>Networks~Network reliability</concept_desc>
	<concept_significance>100</concept_significance>
	</concept>
	</ccs2012>
\end{CCSXML}

\ccsdesc[500]{Computer systems organization~Embedded systems}
\ccsdesc[300]{Computer systems organization~Redundancy}
\ccsdesc{Computer systems organization~Robotics}
\ccsdesc[100]{Networks~Network reliability}

%%
%% Keywords. The author(s) should pick words that accurately describe
%% the work being presented. Separate the keywords with commas.
\keywords{datasets, neural networks, gaze detection, text tagging}

%%
%% This command processes the author and affiliation and title
%% information and builds the first part of the formatted document.
\maketitle

%%
%% Main chapters. Feel free to add as many as you wish.
\section{Introduction}
\label{sec:introduction}

Motivation--> TODO


\section{Background}
\label{sec:background}

we can mention briefly in this section what we have done so far:\\
--client-server architecture\\
--replication and distribution of data\\
--Base properties: basic availability, soft-state, eventual consistency\\
--milestone 4\\
--milestone1, milestone2, milestone3 : should we mention how we got this point, how we developed our system over the milestones??


% !TeX root = ../main.tex

\section{Summary}
\label{sec:summary}

To conclude our paper, we 


%%
%% The next two lines define the bibliography style to be used, and
%% the bibliography file.
\bibliographystyle{ACM-Reference-Format}
\bibliography{bibliography}

%%
%% If your work has an appendix, this is the place to put it.
\appendix


\end{document}
\endinput
